\documentclass[10pt]{article}\nofiles%dans le modele
\newif\ifpdf
	\ifx\pdfoutput\undefined
	\pdffalse % we are not running PDFLaTeX
	\else
	\pdfoutput=1 % we are running PDFLaTeX
	\pdftrue
	\fi

	\ifpdf
	\usepackage[pdftex]{graphicx}
	\else
	\usepackage{graphicx}
	\fi
\usepackage{latexsym}
\usepackage[cmex10]{amsmath}
%%option [cmex10] sinon probl�me dans les small size des delimiteurs
%%n'utilise pas le bon cmex !
\usepackage{amssymb,latexsym}
\usepackage{graphpap}
\usepackage[dvipsnames]{color}
\usepackage{theorem}
\usepackage{alltt}
\usepackage[applemac]{inputenc}
\usepackage[french]{babel}
%\usepackage[francais]{babel}%si on veut juste la cesure francaise
%et pas le style french
%il faut le dernier package french V3,48 dispo
%sur le ftp.univ-rennes1.fr /pub/gut/french !
\usepackage{symbolesmath}
\usepackage{gestion_feuille}
\usepackage{para_perso}
\usepackage{camlindenttt}
	\ifpdf
	\DeclareGraphicsExtensions{.pdf, .jpg}
	\else
	\DeclareGraphicsExtensions{.eps, .jpg}
	\fi
	
	
\solfalse      %avec ou sans solution
\totaltrue  %pour voir tous les exos
\orifalse   %avec origine des exos
\concfalse %avec conncours pour exo d'oral..
\annfalse %avec ann�e ou l'exercice a �t� pos�

\begin{document} 
\centerline{Clavier logique, ensemble (Commande + Espace)}
\begin{tabular}[t]{|c|c|c|c|c|c|c||}
\hline
 & lettre & maj & alt & alt maj & ctrl & ctrl maj\\\hline\hline
 A&application\\  \hline
B&bijection & & & &bijective\\\hline
C&c'est {\`a} dire\\\hline
D&d�finition&&d�monstration&&on d�montre\\\hline
E&ensemble&&sous-ensemble&&il existe\\\hline
F\\\hline
G\\\hline
H\\\hline
I&injection&&& &injective\\\hline
J\\\hline
K\\\hline
L\\\hline
M\\\hline
N&n�cessairement\\\hline
O\\\hline
P&proposition\\\hline
Q&quelconque&&tel que&tels que&telle que&telles que\\\hline
R&r{\'e}currence&respectivement&r{\'e}ciproquement\\\hline
S&si et seulement si\\\hline
T&pour tout&pour tous\\\hline
\end{tabular}

\end{document}
