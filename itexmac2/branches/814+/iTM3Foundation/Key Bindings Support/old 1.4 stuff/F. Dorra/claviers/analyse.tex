\documentclass[10pt]{article}\nofiles%dans le modele
\newif\ifpdf
	\ifx\pdfoutput\undefined
	\pdffalse % we are not running PDFLaTeX
	\else
	\pdfoutput=1 % we are running PDFLaTeX
	\pdftrue
	\fi

	\ifpdf
	\usepackage[pdftex]{graphicx}
	\else
	\usepackage{graphicx}
	\fi
\usepackage{latexsym}
\usepackage{lscape}
\usepackage[cmex10]{amsmath}
%%option [cmex10] sinon probl�me dans les small size des delimiteurs
%%n'utilise pas le bon cmex !
\usepackage{amssymb,latexsym}
\usepackage{graphpap}
\usepackage[dvipsnames]{color}
\usepackage{theorem}
\usepackage{alltt}
\usepackage[applemac]{inputenc}
\usepackage[french]{babel}
%\usepackage[francais]{babel}%si on veut juste la cesure francaise
%et pas le style french
%il faut le dernier package french V3,48 dispo
%sur le ftp.univ-rennes1.fr /pub/gut/french !
\usepackage{gestion_feuille}
\usepackage{para_perso}
\usepackage{camlindenttt}
	\ifpdf
	\DeclareGraphicsExtensions{.pdf, .jpg}
	\else
	\DeclareGraphicsExtensions{.eps, .jpg}
	\fi
	
	
\solfalse      %avec ou sans solution
\totaltrue  %pour voir tous les exos
\orifalse   %avec origine des exos
\concfalse %avec conncours pour exo d'oral..
\annfalse %avec ann�e ou l'exercice a �t� pos�

\begin{document}
\begin{landscape}
\centerline{clavier d'analyse (Commande + $>$)}
\begin{tabular}[t]{|c|c|c|c|c|c|c||}
\hline
 & lettre & maj & alt & alt maj & ctrl & ctrl maj\\\hline\hline
 A&valeur d'adh{\'e}rence&point d'accumulation&&&absolue&absolument\\  \hline
B&\\\hline 
C&converge&convergent&convergence&convergente&&\\\hline
D&divergente&&diff{\'e}rentiable&diff{\'e}rentielle&diff{\'e}omorphisme\\\hline 
E&s{\'e}rie enti{\`e}re&s{\'e}ries enti{\`e}res\\\hline
F&s{\'e}rie de Fourier&s{\'e}ries de Fourier\\\hline
G&\\\hline
H&hom{\'e}omorphisme\\\hline
I&int{\'e}grale&&int{\'e}grable&&IDP (1)\\\hline
J\\\hline
K&compact&\\\hline
L&d{\'e}veloppement limit{\'e}&d{\'e}veloppements limit{\'e}s&&&lipschitzienne\\\hline
M\\\hline
N&espace vectoriel norm{\'e}&espaces vectoriels norm{\'e}s&&&normale&normalement\\\hline
O\\\hline
P&s{\'e}rie {\`a} termes positifs&s{\'e}ries {\`a} termes positifs\\\hline
Q\\\hline
R&rayon de convergence&rayons de convergence\\\hline
S&CSSA (2)&sous-suite&&&simple&simplement\\\hline
T&TCD(3)\\\hline 
U&&&&&uniforme&uniform{\'e}ment\\\hline
V&voisinage \\\hline
W&Bolzano Weierstra${\ss{}}$\\\hline
1&&d{\'e}rivable&&&d{\'e}riv{\'e} \\\hline
0&&continue&&&continuit{\'e} \\\hline
\end{tabular}

(1) int{\'e}grale d{\'e}pendant d'un param{\`e}tre (2) crit{\`e}re sp{\'e}cial des s{\'e}ries alt{\'e}rn{\'e}es (3) th{\'e}or{\`e}me de convergence domin{\'e}e
\end{landscape}
\end{document}
